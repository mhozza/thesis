\chapter*{Úvod}
\addcontentsline{toc}{chapter}{Úvod}
\phantomsection
\markboth{Úvod}{Úvod}

% preco je zarovnanie dolezite
Najnovšie technológie sekvenovania DNA produkujú stále väčšie množstvo sekvencií rôznych organizmov. Spolu s~tým stúpa aj potreba rozumieť týmto dátam.
Dôležitým krokom k~ich porozumeniu je \textit{zarovnávanie sekvencií}.
Zarovnávanie dvoch DNA sekvencií je teda jedným zo základných
bioinformatických problémov. Správne zarovnanie identifikuje časti
sekvencie, ktoré vznikli z~toho istého predka (zarovnané bázy), ako aj
inzercie a delécie v~priebehu evolúcie (medzery v~zarovnaní).
Je nápomocné pri zisťovaní ich štruktúry a následne funkcie jednotlivých častí.

% zarovnavacie algoritmy
Existujú rôzne algoritmy na zarovnávanie sekvencií. Väčšina z~nich je založená na  pravdepodobnostnom modeli, pričom sa snažia nájsť zarovnanie s~čo najväčšou pravdepodobnosťou.
Algoritmy sú zvyčajne založené na dynamickom programovaní a pracujú v~kvadratickom čase v~závislosti od dĺžok sekvencií. Niekedy sa na urýchlenie použijú rôzne heuristické algoritmy, ktoré nie vždy nájdu najpravdepodobnejšie zarovnanie, ale pracujú oveľa rýchlejšie.

My sme sa v~práci zaoberali algoritmom, ktorý hľadá zarovnanie pomocou jednoduchých párových skrytých
Markovovských modelov (pHMM) \cite{durbin}, kde kvalita výsledného zarovnania je ovplyvnená len pravdepodobnostným modelom.

Základný model berie do úvahy len jednotlivé \textit{bázy} a pravdepodobnosti \textit{substitúcie (mutácie)}, \textit{inzercie} a \textit{delécie}.
Náš model navyše uvažuje aj prídavné informácie (takzvané anotácie) získané z~externých programov (napr. anotácie o~génoch z~vyhľadávača génov).

% naco je dobra klasifikacia v zarovnaniach
Keďže množstvo dodatočnej informácie môže byť veľmi veľké -- napríklad pre 3 binárne anotácie by sme mali $2^3 \times 2^3 = 64$ krát väčší počet parametrov --
je ťažké skonštruovať vhodnú skórovaciu maticu pre zarovnávací algoritmus.
Namiesto nej sme teda použili klasifikátory\footnote{program, ktorý na základe vstupnej informácie a vopred natrénovaných parametrov klasifikuje dáta do niektorej triedy z~danej množiny tried}, ktoré  sme trénovali na sekvenciách so známym zarovnaním a potom použili na zarovnanie nových sekvencií. Naše klasifikátory vracajú čísla z~intervalu $\left<0, 1\right>$, ktoré určujú, či dané dve bázy majú byť zarovnané spolu.
% Ked mi bude rozbijat footenote na 2 strany, kuk sem: http://www.tex.ac.uk/cgi-bin/texfaq2html?label=splitfoot

Ako klasifikátor sme použili \emph{náhodný les} \cite{randomForestPaper}, pretože aktuálne patrí medzi najlepšie klasifikátory.

% Ako klasifikátor sme použili \textit{Náhodný les (Random forest)}, poprípade skúsime aj nejaké iné a porovnáme úspešnosť.

%obsah prace
% \todo zmenit podla skutocnosti

% V práci sme sa zaoberali rôznymi spôsobmi trénovania, pričom sme navrhli vlastné modely a porovnáme už s existujúcimi. Ďalej sa budeme zaoberať spôsobom zberu dodatočných informácií, využitu natrénovaného klasifikátora na zarovnávanie nových sekvencií aj pri \textit{prezarovnaní} už existujúcich zarovnaní a implementujeme univerzálny framework, ktorý umožní využívať náš klasifikátor pri rôznych zarovnaniach.

\todo napisat este strucny obsah prace

