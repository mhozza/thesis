\chapter*{Úvod}
\addcontentsline{toc}{chapter}{Úvod}
\phantomsection


% preco je zarovnanie dolezite
Najnovšie technológie sekvenovania DNA produkujú stále väššie monožstvo DNA sekvencií rôznych organizmov. Spolu s tým stúpa aj potreba rozumieť týmto dátam. 
Dôležitým krokom k ich porozumeniu je \textit{zarovnávanie sekvencií}. Pomocou zarovnávania sekvencií dokážeme zistiť ich štruktúru a následne funkciu jednotlivých častí.

% zarovnavacie algoritmy
Existujú rôzne algoritmy na zarovnávanie sekvencií. Väčšina z nich je založená na nejakom pravdepodobostnom modeli, pričom sa snažia nájsť zarovnanie s čo najväčšou pravdepodobnosťou. Či už ide o algoritmy založené na dynamickom programovaní, ktoré s určitosťou nájdu najpravdepodbnejšie zarovnanie, ale za cenu kvadradického času v závislosti od dĺžok sekvencií, alebo heuristické algoritmy, ktoré nie vždy nájdu najpravdepodobnejšie zarovnanie, ale pracujú oveľa rýchlejšie.

My sa budeme zaoberať algoritmom založeným na dynamickom programovaní, kde kvalita výsledného zarovnania je ovplyvnená len pravdepodobnostným modelom.

Základný model berie do úvahy len jednotlivé \textit{bázy} a pravdepodobnosti \textit{substitúcie (mutácie)}, \textit{inzercie} a \textit{delécie}.
Náš model bude navyše uvažovať aj dodatočné informácie získané napríklad z rôznych anotátorov.

% naco je dobra klasifikacia v zarovnaniach
Keďže množstvo dodatočnej informácie môže byť veľmi veľké, %TODO : radovo kolko?
nie je možné skonštruovať vhodnú skórovaciu maticu pre zarovnávací algoritmus.
Namiesto nej teda použijeme klasifikátor, ktorý natrénujeme na sekvenciách so známim zarovnaním a potom použijeme na zarovnanie nových sekvencií. V našom prípade sme si vybrali klasifikátor \textit{Náhodný les (Random forest)}.

%obsah prace
V práci sa budeme zaoberať rôznymi spôsobmi trénovania, pričom navrneme vlastné modely a porovnáme už s existujúcimi. Ďalej sa budeme zaoberať spôsobom zberu dodatočných informácií, využitu natrénivaného klasifikátora na zarovnávanie nových sekvencií aj pri \textit{prezarovnaní} už existujúcich zarovnaní a implementujeme univerzálny framework, ktorý umožní využívať náš klasifikátor pri rôznych zarovnaniach.

