\chapter*{Úvod}
\addcontentsline{toc}{chapter}{Úvod}
\phantomsection
\markboth{Úvod}{Úvod}

% preco je zarovnanie dolezite
Najnovšie technológie sekvenovania DNA produkujú stále väčšie množstvo sekvencií rôznych organizmov. Spolu s~tým stúpa aj potreba rozumieť týmto dátam, čomu napomáha \textit{zarovnávanie sekvencií}.
Zarovnávanie dvoch DNA sekvencií je jedným zo základných
bioinformatických problémov.
DNA sekvencia je reťazec, ktorý sa skladá z jednotlivých báz: adenín (A), guanín (G), cytozín (C) a tymín (T) \cite{wiki:dna}.
Správne zarovnanie identifikuje časti
sekvencie, ktoré vznikli z~toho istého predka (zarovnané bázy), ako aj
inzercie a delécie v~priebehu evolúcie (medzery v~zarovnaní).
Je nápomocné pri zisťovaní ich štruktúry a následne funkcie jednotlivých častí.

% zarovnavacie algoritmy
Existujú rôzne algoritmy na zarovnávanie sekvencií.
% Väčšina z~nich je založená na  pravdepodobnostnom modeli, pričom sa snažia nájsť zarovnanie s~čo najväčšou pravdepodobnosťou.
Algoritmy sú zvyčajne založené na dynamickom programovaní a pracujú v~kvadratickom čase v~závislosti od dĺžok sekvencií. Niekedy sa na urýchlenie použijú rôzne heuristické algoritmy, ktoré nie vždy nájdu najpravdepodobnejšie zarovnanie, ale pracujú oveľa rýchlejšie. Jeden z nich je aj algoritmus na zarovnávanie sekvencií pomocou jednoduchých párových skrytých Markovovských modelov (pHMM) \cite{durbin}, kde kvalita výsledného zarovnania je ovplyvnená len pravdepodobnostným modelom.
Základný model berie do úvahy len jednotlivé bázy a pravdepodobnosti substitúcie (mutácie), inzercie a delécie.

% naco je dobra klasifikacia v zarovnaniach
My sme sa v práci zaoberali návrhom modelov pre tento algoritmus. Naše modely navyše uvažujú aj prídavné informácie (takzvané anotácie) získané z~externých programov (napr. anotácie o~génoch z~vyhľadávača génov). Na zakomponovanie tejto informácie sme sa rozhodli využiť klasifikátory\footnote{klasifikátor je program, ktorý na základe vopred natrénovaných parametrov klasifikuje vstupné dáta do niektorej triedy z~danej množiny tried}, ktoré  sme trénovali na sekvenciách so známym zarovnaním a potom použili na zarovnanie nových sekvencií. Naše klasifikátory vracajú čísla z~intervalu $\left<0, 1\right>$, ktoré určujú, či dané dve bázy majú byť zarovnané spolu.

% Keďže množstvo dodatočnej informácie môže byť veľmi veľké -- napríklad pre 3 binárne anotácie by sme mali $2^3 \times 2^3 = 64$ krát väčší počet parametrov --
% je ťažké skonštruovať vhodnú skórovaciu maticu pre zarovnávací algoritmus.
% Namiesto nej sme teda použili klasifikátory

% Ked mi bude rozbijat footenote na 2 strany, kuk sem: http://www.tex.ac.uk/cgi-bin/texfaq2html?label=splitfoot

Ako klasifikátor sme použili \emph{náhodný les} \cite{randomForestPaper}, pretože aktuálne patrí medzi najlepšie klasifikátory.

%obsah prace

V kapitole \ref{chap:alignment} popíšeme algoritmy na zarovnávanie a modely, z ktorých budeme vychádzať. V kapitole \ref{chap:other-work} si predstavíme súvisiace práce. Potom si predstavíme naše riešenie, ktoré sme v našej práci rozdelili do dvoch častí. V kapitole \ref{chap:clf} si povieme niečo o náhodných lesoch a predstavíme si klasifikátory, ktoré použijeme v našich modeloch. Povieme si niečo voľbe atribútov a vlastnostiach klasifikátora. V kapitole \ref{chap:models} si predstavíme dva modely s klasifkátorom na zarovnávanie sekvencií s dodatočnou informáciou a porovnáme ich úspešnosť s referenčnými modelmi. Nakoniec si v kapitole \ref{chap:implementation} povieme niečo o tom ako sme implementovali náš zarovnávač.
