\chapter*{Úvod}
\addcontentsline{toc}{chapter}{Úvod}
\phantomsection
\markboth{Úvod}{Úvod}

% preco je zarovnanie dolezite
Najnovšie technológie sekvenovania DNA produkujú stále väčšie monožstvo DNA sekvencií rôznych organizmov. Spolu s tým stúpa aj potreba rozumieť týmto dátam.
Dôležitým krokom k ich porozumeniu je \textit{zarovnávanie sekvencií} -- nájdenie podobností medzi sekvenciami.
Zarovnávanie sekvencií je nápomocné pri zisťovaní ich štruktúry a následne funkciu jednotlivých častí.

% zarovnavacie algoritmy
Existujú rôzne algoritmy na zarovnávanie sekvencií. Väčšina z nich je založená na  pravdepodobostnom modeli, pričom sa snažia nájsť zarovnanie s čo najväčšou pravdepodobnosťou.
Algoritmy sú zvyčajne založené na dynamickom programovaní. Sú buď deterministické, ktoré s určitosťou nájdu najpravdepodbnejšie zarovnanie, ale za cenu kvadradického času v závislosti od dĺžok sekvencií, alebo heuristické algoritmy, ktoré nie vždy nájdu najpravdepodobnejšie zarovnanie, ale pracujú oveľa rýchlejšie.

My sa budeme zaoberať algoritmom založeným na dynamickom programovaní, kde kvalita výsledného zarovnania je ovplyvnená len pravdepodobnostným modelom.

Základný model berie do úvahy len jednotlivé \textit{bázy} a pravdepodobnosti \textit{substitúcie (mutácie)}, \textit{inzercie} a \textit{delécie}.
Náš model bude navyše uvažovať aj dodatočné informácie získané napríklad z rôznych anotátorov.

% naco je dobra klasifikacia v zarovnaniach
Keďže množstvo dodatočnej informácie môže byť veľmi veľké -- napríklad pre 3 binárne anotátory by sme mali $2^3 \times 2^3 = 64$ krát väčší počet parametrov --
je ťažké skonštruovať vhodnú skórovaciu maticu pre zarovnávací algoritmus.
Namiesto nej teda použijeme klasifikátor, ktorý natrénujeme na sekvenciách so známim zarovnaním a potom použijeme na zarovnanie nových sekvencií. Klasifikátor bude vracať pravdepodobnosť, že dané dve bázy majú byť zarovnané spolu.
Ako klasifikátor použijeme \textit{Náhodný les (Random forest)}, poprípade skúsime aj nejaké iné a porovnáme úspešnosť.

%obsah prace
\todo zmenit podla skutocnosti

V práci sa budeme zaoberať rôznymi spôsobmi trénovania, pričom navrneme vlastné modely a porovnáme už s existujúcimi. Ďalej sa budeme zaoberať spôsobom zberu dodatočných informácií, využitu natrénovaného klasifikátora na zarovnávanie nových sekvencií aj pri \textit{prezarovnaní} už existujúcich zarovnaní a implementujeme univerzálny framework, ktorý umožní využívať náš klasifikátor pri rôznych zarovnaniach.

