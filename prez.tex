%Ohurit komisiu
%Urobili vela prace, Nebolo to lahke

\documentclass[xcolor=dvipsnames, compress]{beamer} 
\usepackage[utf8]{inputenc}
\usepackage[slovak]{babel}
\usepackage{lmodern}
\usepackage[T1]{fontenc}
\usepackage{amsfonts}
\usepackage{amssymb}
\usepackage{amsthm}
\usepackage{amsmath}
\usepackage{epsfig}
\usepackage{wrapfig}
\usepackage{caption}
\usepackage{subcaption}
\usepackage{url}
\usepackage{hyperref}
\usepackage{multicol}
\usecolortheme[named=Green]{structure} 
\usetheme{Dresden} 
% \setbeamertemplate{items}[ball] 
% \setbeamertemplate{blocks}[rounded][shadow=true] 

% \useoutertheme{umbcfootline} 
% \AtBeginSection[]{
% \frame<beamer>{
%   \begin{multicols}{2}
%     \ifpdf
%       \pdfbookmark[0]{Contents}{toc}
%     \fi    
%     \frametitle{Obsah}
%     \setcounter{tocdepth}{2}      
%     \tableofcontents[currentsection,subsections,hideothersubsections]
%   \end{multicols} 
% }
% }


% items enclosed in square brackets are optional; explanation below
\title{Zarovnávanie sekvencií s použitím metód klasifikácie}
\subtitle{
\vspace{0.5cm}
\small Diplomová práca
}
\author[Michal Hozza]{Bc. Michal Hozza \\ Vedúci práce: Mgr. Tomáš Vinář, PhD. \\ Konzultant: Mgr. Michal Nanási}
\institute[FMFI UK]{
  Fakulta matematiky, fyziky a informatiky,
  Univerzita Komenského, Bratislava\\
}
\date{\today}

\begin{document}

%--- the titlepage frame -------------------------%
\begin{frame}[plain]
  \titlepage
\end{frame}

%--- the presentation begins here ----------------%

% \begin{frame}{Obsah prezentácie}
%   % \begin{itemize}
%   % \item Cieľ
%   % \item Zarovnávanie sekvencií
%   % \item 
%   % \end{itemize} 

% \end{frame}
\section{Cieľ}
\begin{frame}{Cieľ}
  \begin{itemize}
  \item Cieľom práce je vytvoriť nové metódy na korekciu zarovnaní biologických sekvencií na základe prídavnej informácie.
  \item Integrácia tejto informácie bude zabezpečená pomocou techník využívaných na klasifikáciu v strojovom učení.
  \end{itemize} 
\end{frame}


\section{Zarovnávanie sekvencií}
\begin{frame}{Zarovnávanie sekvencií}
Kľúčové problémy:
  \begin{itemize}
    \item Aké typy zarovnávania by sme mali uvažovať
    \item Skórovací systém, ktorý použijeme na ohodnotenie zarovnania a trénovanie
    \item Algoritmus, ktorý použijeme na hľadanie optimálneho alebo dobrého zarovnania podľa skórovacieho systému
    \item Štatistická významnosť zarovnania.
  \end{itemize} 
\end{frame}

\section[Modely]{Generatívny vs. Diskriminačný model}
\begin{frame}{Generatívny vs. Diskriminačný model}
Generatívny:
\begin{itemize}
\item sa snaží modelovať proces, ktorý generuje dáta ako pravdepodobnosť P (X, Y, Z)
\item rozložíme ju pomocou nezávislých predpokladov na procese -> obmedzujúce
\end{itemize} 
Diskriminačný
\begin{itemize}
\item priamo odhaduje P(Z|X,Y) alebo prislúchajúcu diskriminačnú funkciu, a preto sa zamerá na podstatnú časť problému odhadu
\item Nepotrebuje nezávislosť -> silnejšie
\end{itemize}   
\end{frame}

% Existujúce riešenia
% Problém inverzného zarovnania
% Support vector training of protein alignment models
% Support Vector Machine (SVM)
% Umožňuje trénovať pomocou rôznych účelových funkcií
% Contralign: Discriminative training for protein sequence alignment.
% Conditional Rnadom Fields (CRF)
% Neumožnuje trénovať pomocou rôznych účových funkcií

% Odlišnosti nášho riešenia
% Rôzne metódy trénovania
% Možnosť učenia bez učiteľa
% Iný klasifikátor (možno viac rôznych klasifikátorov, pípadne abstrakcia od klasifikátora)

% Random Forest
% Klasifikátor
% Zložený z klasifikačných (rozhodovacích) stromov, ktoré hlasujú

% Simulátor
% Program, ktorý simuluje evolúciu
% Model určený na prvotné experimenty


% \section{Zarovnávanie sekvencií}
% \begin{frame}{Zarovnávanie sekvencií}
% Kľúčové problémy:
%   \begin{itemize}
%     \item Aké typy zarovnávania by sme mali uvažovať
%     \item Skórovací systém, ktorý použijeme na ohodnotenie zarovnania a trénovanie
%     \item Algoritmus, ktorý použijeme na hľadanie optimálneho alebo dobrého zarovnania podľa skórovacieho systému
%     \item Štatistická významnosť zarovnania.
%   \end{itemize} 
% \end{frame}

% \section{Zarovnávanie sekvencií}
% \begin{frame}{Zarovnávanie sekvencií}
% Kľúčové problémy:
%   \begin{itemize}
%     \item Aké typy zarovnávania by sme mali uvažovať
%     \item Skórovací systém, ktorý použijeme na ohodnotenie zarovnania a trénovanie
%     \item Algoritmus, ktorý použijeme na hľadanie optimálneho alebo dobrého zarovnania podľa skórovacieho systému
%     \item Štatistická významnosť zarovnania.
%   \end{itemize} 
% \end{frame}

% \section{Zarovnávanie sekvencií}
% \begin{frame}{Zarovnávanie sekvencií}
% Kľúčové problémy:
%   \begin{itemize}
%     \item Aké typy zarovnávania by sme mali uvažovať
%     \item Skórovací systém, ktorý použijeme na ohodnotenie zarovnania a trénovanie
%     \item Algoritmus, ktorý použijeme na hľadanie optimálneho alebo dobrého zarovnania podľa skórovacieho systému
%     \item Štatistická významnosť zarovnania.
%   \end{itemize} 
% \end{frame}



% \begin{frame}{Ďakujem za pozornosť!}
%   \begin{center}
% {\bf Ďakujem za pozornosť!} 
%   \end{center}
  
%   \vspace{3cm}
  
%   \begin{center}
%   \small{Ospravedlňujem sa za kvalitu obrázkov.}
%   \end{center}
% \end{frame}


\end{document}
