Zarovnávanie dvoch DNA sekvencií je jedným zo základných
bioinformatických problémov. Obvykle takéto zarovnanie hľadáme pomocou jednoduchých párových skrytých
Markovovských modelov (pHMM). V tejto práci sa zaoberáme možnosťami použitia prídavnej informácie o funkcii vstupných sekvencií na zlepšenie kvality takýchto zarovnaní.
Informácie sme zakomponovali pomocou klasifikátorov, ktoré rozhodujú či dané pozície majú byť zarovnané k sebe alebo nie. Ako klasifikátor sme použili Random forest.

Ukázalo sa, že klasifikátor sa dokáže naučiť, ktoré okná majú byť zarovnané k sebe a ktoré nie.

Vyvinuli sme 2 modely pre zarovnanie sekvencií s anotáciami za pomoci klasifikátora, ktoré sú založené na párových skrytých Markovovských modeloch.

Model s klasifikátorom ako emisiou, kde sme nahradili emisné tabuľky stavov výstupom z klasifikátora.

Model s klasifikátorovou páskou, kde navyše modelujeme aj výstup z klasifikátora.
\\ \\
{\bf Kľúčové slová:} zarovnávanie sekvencií, strojové učenie, Random forest, anotácie
