Zarovnávanie dvoch DNA sekvencií je jedným zo základných
bioinformatických problémov. V~tejto práci sa zaoberáme možnosťami použitia prídavnej informácie o~funkcii vstupných sekvencií na zlepšenie kvality takýchto zarovnaní.
Informácie sme zakomponovali pomocou dvoch klasifikátorov, jeden pre zarovnané časti sekvencií a druhý pre nezarovnané časti sekvencií.
Klasifikátor rozdeľuje pozície do dvoch tried: tie, ktoré majú byť zarovnané k~sebe (trieda 1) a tie, ktoré nie (trieda 0). V~prípade klasifikátora pre nezarovnané časti, tie, ktoré majú byť zarovnané k~medzere (trieda 1) a tie ktoré nie (trieda 0). Výstupom klasifikátora je potom pravdepodobnosť, že ďáta patria do triedy 1. Na klasifikáciu sme použili náhodné lesy (Random Forests). Venovali sme sa výberu atribútov a vhodnými atribútmi sa nám podarilo zlepšiť úspešnosť klasifikátorov. Ukázali sme, že klasifikátor sa dokáže naučiť, ktoré pozície majú byť zarovnané k~sebe a ktoré nie.

Vyvinuli sme dva modely pre zarovnanie sekvencií s~anotáciami za pomoci klasifikátora, ktoré sú založené na párových skrytých Markovovských modeloch.
V~modeli A~sme nahradili emisné tabuľky stavov výstupom z~klasifikátora.
V~modeli B modelujeme okrem báz aj pásku s~výstupom z~klasifikátora.
Naše modely dokázali prekonať referenčné modely na biologických dátach aj na simulovaných dátach s~vyššou dôležitosťou anotácie. Na simulovaných dátach dosiahol model B podobné výsledky a model A~mierne horšie.
\\ \\
{\bf Kľúčové slová:} zarovnávanie sekvencií, dodatočná informácia, strojové učenie, náhodné lesy, anotácie, skryté markvovské modely
