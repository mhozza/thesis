\chapter{Random forest[Draft]}

V tejto kapitole si popíšeme ako funguje klasifikátor \textit{Random forest}, v čom spočívajú jeho výhody a prečo sme si ho vybrali.

\section{Klasifikačné stromy}
Keďže Random forest je vlastne les \textit{klasifikačných stromov (niekedy označované ako rozhodovacie stromy)}, je nevyhnutné, aby sme si najskôr povedali niečo o nich.

%section?
Klasifikačný strom je klasifikátor vybudovaný na základe trénovacej množiny, ktorý na základe vstupných údajov (vstupného vektora) predpovedá hodnotu výstupnej premennej. Klasifikátor sa skladá z \textit{uzlov} a \textit{listov}. V uzle sa nachádza rozdeľovacie kritérium, podľa ktorého sa vieme rozhodnúť do ktorej vetvy máme pokračovať. V listoch sa nachádzajú triedy, do ktorých sa vstupný vektor klasifikuje.
%obrazok

Klasifikácia vstupného vektora vyzerá nasledovne: prechádzame strom zhora na dol a postupne sa zaraďujeme do vetiev podľa rodzelovacích kritérií. Podľa toho, v ktorom liste skončíme, sa určí výstupná hodnota.

\subsection{Trénovanie}
TODO - strucne popisat rozne metody + odkazat na literaturu

\section{Random forest}

\subsection{Klasifikácia}
Random forest sa skladá z mnohých klasifikačných stromov. Pri klasifikácii nejakého vstupnéeho vektora sa predloží vstupný vektor každému stromu. Každý strom následne vráti klasifikáciu a hovoríme, že stromy \textit{hlasujú}. Random forest následne zvolí klasifikáciu podľa väčšiny z hlasov všetkých stromov v lese.
%TODO parametrizacia rf

\subsection{Trénovanie}
Každý strom sa buduje nasledovne.
Máme $N$ trénovacích vektorov, pričom každý vektor sa skladá z $M$ vstupných premenných (teda má rozmer $M$). 
Z nich vyberieme \textit{náhodne s opakovaním} $N$ vektorov, ktoré budú použité ako tréenovacia množina pre daný rozhodovací strom.
Okrem toho vyberieme náhodne $m<<M$ vstupných parametrov a len tie sa použijú pri trénovaní stromu. Každý strom je vybudovaný najviac ako sa dá - nie je žiadne orezávanie.

Chyba RF závisí od dvoch vecí
\begin{itemize}
\item korelácia
\item sila
\end{itemize}


\section[Dôležité vlastnosti]{Dôležité vlastnosti Random forest-u}
\begin{itemize}
\item Je najpresnejší spomedzi všetkých terajších klasifikačných algoritmov.
\item Je efektívny aj na veľkých dátach.
\item Dokáže obsiahnuť tisícky vstupných premenných bez ich vymazávania.
\item Dáva odhad, ktoré premenné sú dôležité na klasifikáciu.
\item It generates an internal unbiased estimate of the generalization error as the forest building progresses.
\item It has an effective method for estimating missing data and maintains accuracy when a large proportion of the data are missing.
\item It has methods for balancing error in class population unbalanced data sets.
\item Generated forests can be saved for future use on other data.
\item Prototypes are computed that give information about the relation between the variables and the classification.
\item Počíta blízkosť medzi jednotlivími pármi prípadov, čo môže byť použité pri clusteringu a detekcii outlierov alebo ponúknuť zaujímavé pohľady na dáta.
\item Predchádzajúce môže byť rozšírené aj na neolabelované dáta (bez informácie o triede), čo sa dá použiť na klasifikáciu, pohľady na dáta a detekciu outlierov bez učiteľa.
\item Ponúka experimentálnu metódu na detekciu interakcie medzi premennými.
\end{itemize}



