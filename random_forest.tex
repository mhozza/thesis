\chapter{Random forest}

V tejto kapitole si popíšeme ako funguje klasifikátor \textit{Random forest}, v čom spočívajú jeho výhody a prečo sme si ho vybrali.

\section{Klasifikačné stromy}
Keďže Random forest je vlastne les \textit{klasifikačných stromov (niekedy označované ako rozhodovacie stromy)}, je nevyhnutné, aby sme si najskôr povedali niečo o nich.

%section?
Klasifikačný strom je klasifikátor vybudovaný na základe trénovacej množiny, ktorý na základe vstupných údajov (vstupného vektora) predpovedá hodnotu výstupnej premennej. Klasifikátor sa skladá z \textit{uzlov} a \textit{listov}. V uzle sa nachádza rozdeľovacie kritérium, podľa ktorého sa vieme rozhodnúť do ktorej vetvy máme pokračovať. V listoch sa nachádzajú triedy, do ktorých sa vstupný vektor klasifikuje.
%obrazok

Klasifikácia vstupného vektora vyzerá nasledovne: prechádzame strom zhora na dol a postupne sa zaraďujeme do vetiev podľa rodzelovacích kritérií. Podľa toho, v ktorom liste skončíme, sa určí výstupná hodnota.


\section{Random forest}

\subsection{Klasifikácia}
Random forest sa skladá z mnohých klasifikačných stromov. Pri klasifikácii nejakého vstupnéeho vektora sa predloží vstupný vektor každému stromu. Každý strom následne vráti klasifikáciu a hovoríme, že stromy \textit{hlasujú}. Random forest následne zvolí klasifikáciu podľa väčšiny z hlasov všetkých stromov v lese.
%TODO parametrizacia rf

\subsection{Trénovanie}
Každý strom sa buduje nasledovne.
Máme $N$ trénovacích vektorov, pričom každý vektor sa skladá z $M$ vstupných premenných (teda má rozmer $M$). 
Z nich vyberieme \textit{náhodne s opakovaním} $N$ vektorov, ktoré budú použité ako tréenovacia množina pre daný rozhodovací strom.
Okrem toho vyberieme náhodne $m<<M$ vstupných parametrov.

\section{Dôležité vlastnosti Random forest-u}




