\chapter{Implementácia}

\section{Simulátor}

Simulátor sĺuži na overenie funkčnosti klasifikátora. Náhodne vygeneruje 2 sekvencie, ktoré vznikli zo spoločného predka a vyrobí korektné zarovnanie. Okrem toho vyrobí aj nejaké dodatočné informácie ktoré majú pomôcť pri zarovnávaní.

\subsection{Algoritmus}
Simulátor vygeneruje informáciu o tom, ktoré časti sekvencie prisĺuchaju génom a ktoré nie. Informácia má podobu boolovskeho vektora.
Simulátor najskôr vygeneruje \textit{základnú (master) postupnosť} a z nej odvodí dve ďalšie postupnosti, ktoré zodpovedajú sekvenciám.

Okrem toho simulátor vygeneruje dve sekvencie, pričom prvú vyrobí náhodne a druhú odvodí z nej pomocou mutácie a inzercie/delécie.
V našom prípade sme inzerciu vynechali a simulujeme ju ako deléciu v druhej sekvencii.
Pri odvodzovaní bude používať aj informáciu o tom, ktorá časť je gén a ktorá nie.

Keďže simulátor vie spôsob akým generoval druhú sekvenciu z prvej, vie aj korektné zarovnanie.

Simulátor má vopred daných niekolko konštánt -- pravdepodobnosti udalostí, ktoré môžu nastať.
%TODO: ak bude treba zaberat miesto, drbnut sem tabuku

\subsection{Generovanie informácie o génoch}

\subsection{Simulácia mutácie}

\subsection{Simulácia delécie}

