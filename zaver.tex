\chapter*{Záver}

V tejto práci sme sa zaoberali zakomponovaním dodatočnej informácie do zarovnávania sekvencií. Dodatočnú informáciu sme mali vo forme anotácií k sekvenciám a informáciu z anotácií sme sa rozhodli zahrnúť pomocou klasifikátorov.

Na začiatku práce sme zhrnuli základné poznatky z oblasti zarovnávania sekvencií, popísali sme si rôzne algoritmy na zarovnávanie sekvencií, vrátane zarovnávania pomocou skrytých Markovovských modelov, ktoré sme zobrali ako základ pre náš zarovnávač.

Potom sme uviedli naše riešenie, ktoré sme rozdelili na dve časti.
V prvej časti sme sa venovali výberu klasifikátorov a ich atribútov, popísali sme si ich vlastnosti a zhodnotili úspešnosť.
Popísali sme si algoritmus klasifikátora náhodné lesy, ktorý sme si vybrali pre použitie v tejto práci. Vyzdvihli sme aj jeho vlastnosti a dôvody, prečo sme si vybrali práve tento klasifikátor.
Predstavili sme si dva typy klasifikátorov: Match klasifikátor a Indel klasifikátor. Zaviedli sme pojem okna v sekvenciách, ktoré obsahovalo okolie daných pozícií, pričom sme uviedli variantu okna pre oba typy klasifikátorov.
Ukázali sme spôsob trénovania jednotlivých klasifikátorov a výber pozitívnych a negatívnych príkladov na trénovanie.
Ďalej sme sa venovali výberu atribútom a podarilo sa nám základné atribúty rozšíriť tak, aby sme zvýšili úspešnosť. Ukázali sme tiež, že voľba typu atribútov je nezávislá na veľkosti okna. Pre okno veľkosti 9 sa nám podarilo klasifikátory natrénovať s úspešnosťou 89,87\% pri Match a 81,78\% pri Indel klasifikátore. Venovali sme sa aj dôležitosti atribútov, pričom sme pozerali na to, ktoré atribúty sú pre klasifikátor najdôležitejšie a ktoré sú najmenej dôležité. Zistili sme, že pre klasifikátor sú najdôležitejšie zhody medzi bázami a bázy samotné. Anotácie a zhody medzi nimi neboli až tak podstatné.

V druhej časti práce sme sa venovali zakomponovaniu klasifikátorov do modelov pre zarovnávanie. Uviedli sme pojem pásky klasifikátora, ktorá obsahuje výstup klasifikátora pre jednotlivé pozície v zarovnaní. Predstavili sme si dva modely: model A a model B.
Model A bol inšpirovaný pHMM, ale nebol pravdepodobnostný, emisné pravdepodobnosti sme v ňom nahradili priamo výstupom z klasifikátorov,
Model B bol stále pHMM, ale navyše modeloval aj klasifikátorovú pásku.
Pri modeli B sme museli vyriešiť problém, že obyčajné HMM pracujú len s diskrétnymi hodnotami, ale výstup z klasifikátora je spojitý. Na to sme uviedli dve riešenia: diskretizáciu výstupu klasifikátora a použitie spojitého HMM.
Ukázali sme, že diskretizácia výstupu klasifikátora je lepšie riešenie a že je vhodné použiť vyhladenie pomocou gausiánu, čo nám umožní vyhladiť šum a doplniť chýbajúce dáta. Popísali sme si spôsob trénovania a porovnali sme naše modely s existujúcimi na rôznych dátových množinách.
Na biologických dátach aj na simulovaných dátach s vyššou dôležitosťou anotácie sa nám podarilo prekonať existujúce riešenia oboma modelmi. Na obyčajných simulovaných dátach model B dosahoval podobnú úspešnosť ako referenčné modely a model A mierne horšiu. Podarilo sa nám teda ukázať, že naše modely sa dajú použiť na zarovnávanie biologických DNA sekvencií s dodatočnou informáciou. Ukázali sme tiež, že použitie anotácie má pozitívny vplyv na kvalitu zarovnania. Okrem toho sme ukázali, že použitie dvoch špecializovaných klasifikátorov pre dva typy stavov je lepšie, ako použiť jeden univerzálny klasifikátor pre oba typy stavov.

Nakoniec sme sa venovali implementácii nášho riešenia, pričom sme spomenuli aké technológie a prečo sme si ich vybrali. Spomenuli sme aj niektoré detaily implementácie, spôsob použitia a možnosti rozšírenia aplikácie.

V budúcnosti je prácu možné rozšíriť niekoľkými spôsobmi. Dá sa vyskúšať použitie iných klasifikátorov, napríklad neurónových sietí so špecializovanou architektúrov. Pri modeli A sa namiesto HMM sa môže použiť aj CRF, pričom by sme dostali pravdepodobnostný model. Dá sa použiť aj učenie bez učiteľa, kde by sa najskôr sekvencie zarovnávali niektorým existujúcim zarovnávačom a potom by sa iteratívne model natrénoval z aktuálneho zarovnania a potom by sa prezarovnal a toto by sa opakovalo až do konvergencie. Umožnilo by to zarovnanie sekvencií nových organizmov, ku ktorým ešte nemáme trénovacie sekvencie. Okrem toho by sa dalo použiť aj klastrovanie dát, kde výsledok klastrovania by bol poskytnutý na páske namiesto výstupu klasifikátora, alebo by mohol byť zahrnutý ako ďalšia anotácia.
