\chapter{Súvisiaca práca}
\label{chap:other-work}

V~tejto kapitole si uvedieme stručný prehľad modelov, ktoré zahŕňajú doplnkové informácie do zarovnania pomocou metód klasifikácie a stručne uvedieme v~čom sa bude náš model líšiť.

V~princípe môžme rozlišovať dva typy modelov - \textit{generatívny model} a \textit{diskriminačný model}.
%Pri generatívnom modeli sa snažíme modelovať ako boli dáta generované. Na základe predpokladov o generovaní sa sažíme určiť triedu.
%Pri diskriminatívnom modeli sa nestaráme o to ako boli dáta generované, jednoducho kategorizuje vstupný vektor. \cite{gendiscmodels}

Konvenčné techniky odhadu pre zarovnávania sa zakladajú na generatívnom modeli. Generatívny model (napr. HMM) sa snaží modelovať proces, ktorý generuje dáta ako pravdepodobnosť $P(X,Y,Z)$, kde $X = x_1x_2\dots x_n$, $Y = y_1y_2\dots y_m$ a Z~je zarovnanie. Ak vieme vypočítať (alebo dobre odhadnúť) $P(X,Y,Z)$, hľadáme
$$\operatorname{ arg\,max}_z P(X = x,Y = y,Z = z)$$
čiže optimálne zarovnanie $z$ z~dvoch sekvencií $x$ a $y$.

Aby sme zľahčili odhad pravdepodobnosti $P(X,Y,Z)$, rozložíme ju pomocou nezávislých pravdepodobností (napríkald v HMM na pravdeodobnosti prechodov a emisné pravdepodobnosti v stavoch).
To síce vedie k~efektívnym a jednoduchým výpočtom odhadu, ale obmedzuje to interakcie v~rámci sekvencií, ktoré by sme mohli modelovať. Napríklad nemôžme modelovať rôzne doplnkové informácie, ktoré sú závislé na sekvenciách \cite{svmTrainingProteinsAlignment}.

Naproti tomu, v diskriminačných modeloch počítame
$$\operatorname{ arg\,max}_z P(Z = z|X = x,Y = y).$$
V tom prípade nemusíme modelovať proces, ktorý generuje $x$ a $y$ a môžme sa zamerať na podstatnú časť problému \cite{svmTrainingProteinsAlignment}.

% Výskum v~oblasti strojového učenia dokázal, že diskriminačné učenie (SVM, RandomForest) zvyčajne produkuje oveľa presnejšie pravidlá ako generatívne učenie (HMM, naive Bayes classifier) \cite{svmTrainingProteinsAlignment}.
% Môže to byť vysvetlené tým, že $P(Z|X,Y)$, je už vhodné na vyhodnotenie optimálnej predikcie
% $$\operatorname{ arg\,max}_z P(Z = z|X = x,Y = y).$$
% \cite{svmTrainingProteinsAlignment}\\
% Diskriminačné učenie aplikované na problém zarovnania bude priamo odhadovať $P(Z|X,Y)$ alebo prislúchajúcu diskriminačnú funkciu, a preto sa zamerá na podstatnú časť problému odhadu \cite{svmTrainingProteinsAlignment}.

Aktuálne existuje len niekoľko prístupov k~diskriminačnému učeniu modelov zarovnávaní.
Jeden z~možných prístupov je riešiť \textit{problém inverzného zarovnania} pomocou strojového učenia. \cite{svmTrainingProteinsAlignment}

\begin{df}[Inverzné zarovnanie]
Máme dané sekvencie a k~nim zarovnanie. Inverzné zarovnanie nám vráti váhový model, s~ktorým daný algoritmus na zarovnávanie vráti požadované zarovnanie k~daným sekvenciám.
\end{df}

Problém inverzného zarovnania bol prvý krát formulovaný v~\cite{inverseAlign1}.
Na tomto probléme je postavený aj model v~\cite{svmTrainingProteinsAlignment}, kde sa na trénovanie Support Vector Machine (SVM) dá pozerať ako na riešenie tohto problému. V~článku sa zaoberajú použitím \textit{Structural SVM} algoritmu na zarovnávanie proteínových sekvencií. Diskriminačné učenie umožňuje zahrnutie množstva dodatočnej informácie -- státisíce parametrov.
Navyše SVM umožňuje trénovanie pomocou rôznych účelových funkcií (loss functions).
SVM algoritmus má lepšiu úspešnosť ako generatívna metóda SSALN, ktorá je veľmi presným generatívnym modelom zarovnaní zahŕňajúcim informáciu o~štruktúre.

Podobný prístup je aj v~CONTRAlign \cite{contralign}, kde sa používajú Conditional Random Fields (CRF). Tento prístup tiež ťaží z~benefitov diskriminačného učenia, avšak na rozdiel od \cite{svmTrainingProteinsAlignment} neumožňuje použitie účelových funkcií.
