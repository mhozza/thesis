\chapter{Zarovnávanie sekvencií}
\section{Podobnosť sekvencií a sekvenčná homológia}
V prírode vznikajú nové sekvencie modifikáciou už existujúcich. Preto môžme často spozorovať podobnosť medzi neznámou sekvenciou a sekvenciou o ktorej už niečo vieme. Ak zistíme podobnosti medzi sekvenciami, môžeme preniesť informácie o štruktúre a/alebo funkcii na novú sekvenciu.

Hovoríme, že dve súvisiace sekvencie sú \textit{homologické}, a že prenášame informácie \textit{podľa homológie}.

Na prvý pohľad rozhodnutie podobnosti dvoch biologických sekvencií nie je nič iné, ako rozhodnutie podobnosti dvoch textových reťazcov. Mnoho metód analýzy biologických sekvencií je preto zakorenená v informatike, kde je už mnoho literatúry týkajúcej sa tejto problematiky.

Vývoj sekvencií hromadí \textit{inzercie}, \textit{delécie} a \textit{substitúcie}, takže predtým ako môže byť vyhodnotená podobnosť, treba urobiť zarovnanie sekvencií. Preto je zarovnanie sekvencií veľmi dôležité.

\section{Skórovacie schémy}
Takmer všetky metódy zarovnania hľadajú zarovnanie dvoch reťazcov na základe nejakej \textit{skórovacej schémy}. Skórovacie schémy môžu byť veľmi jednoduché, napr. $+1$ za \textit{zhodu} a $-1$ za \textit{nezhodu}. Hoci ak chceme mať schému, kde biologicky najkorektnjšie zarovnanie má najvyššie skóre, musíme vziať do úvahy, že biologické sekvencie majú evolučnú históriu, $3D$ štrukturu a mnohé ďalšie vlastnosti obmedzujúce ich primárnu evolúciu. Preto skórovací systém vyžaduje starostlivé premyslenie a môže byť veľmi zložitý.

\section{Párové zarovnávanie}
Párové zarovnávanie je základná úloha zarovnávania sekvencií, kde sa k sebe zarovnávajú dve sekvencie. V tejto práci sa budeme zaoberať len párovým zarovnávaním.

Kľúčové problémy sú:
\begin{enumerate}
\item Aké typy zarovnávania by sme mali uvažovať
\item Skórovací systém, ktorý použijeme na ohodnotenie zarovnania
\item Algoritmus, ktorý použijeme na hľadanie optimálneho alebo dobrého zarovnania podľa skórovacieho systému
\item Štatistické metódy na vyhodnotenie významnosti skóre zarovnania.
\end{enumerate}

\subsection{Typy zarovnaní}
Základné typy zarovnaní sú \textit{Globálne zarovnanie} a \textit{Lokálne zarovnanie}. 
Pri globálnom zarovnaní je výstupom zarovnanie celých sekvencií s najvyšším skóre. 
Pri lokálnomn zarovnaní je výstupom zarovnanie nejakých poreťazcov sekvencií s najvyšším skóre.

%\subsection{Skórovacie systémy}

\subsection{Algoritmy na hľadanie zarovnaní}
Máme daný skórovací systém, potrebujeme algoritmus, ktorý nájde optimálne zarovnanie dvoch sekvencií.
Budeme uvažovať zarovnávanie s medzerami. To znamená, že môžeme do sekvencie pridať ľubovoľne veľa medzier, aby sme dosiahli lepšie skóre. Existuje 
$$ {2n \choose n}  = \frac{(2n)!}{(n!)^2} \simeq \frac{2^{2n}}{\sqrt{\pi n}} $$
možných globálnych zarovnaní. Čiže nie je možné v rozumnom čase nimi prejsť.

Algoritmy na hľadanie zarovnaní využívajú \textit{dynamické programovanie}. Algoritmy s dynamickým programovaním garantujú nájdenie optimálneho zarovnania.
Existujú aj heuristické algoritmy, ktoré môžu byť veľmi rýchle, avšak majú určité predpoklady a môže sa stať, že nenájdu najlepšie zarovnanie pre niektoré páry sekvencií.
My sa budeme zaoberať len algoritmami využívajúcimi dynamické programovanie. Pre rôzne typy zarovnaní máme rôzne algoritmy zarovnávania.

\subsection{Algoritmus pre globálne zarovnanie: Needelman-Wunch}

\subsection{Algoritmus pre lokálne zarovnanie: Smith-Waterman}