\chapter{Zarovnávanie sekvencií}

V tejto kapitole si stručne popíšeme čo je to globálne a lokálne zarovnanie a ukážeme základné algoritmy na hľadanie globálneho a lokálneho zarovnania. Tieto algoritmy budeme neskôr s miernou modifikáciou používať pri našom riešení.

\section{Podobnosť sekvencií a sekvenčná homológia}
V prírode vznikajú nové sekvencie modifikáciou už existujúcich. Preto môžme často spozorovať podobnosť medzi neznámou sekvenciou a sekvenciou o ktorej už niečo vieme. Ak zistíme podobnosti medzi sekvenciami, môžeme preniesť informácie o štruktúre a/alebo funkcii na novú sekvenciu.

Hovoríme, že dve súvisiace sekvencie sú \textit{homologické}, a že prenášame informácie \textit{podľa homológie}.

Na prvý pohľad rozhodnutie podobnosti dvoch biologických sekvencií nie je nič iné, ako rozhodnutie podobnosti dvoch textových reťazcov. Mnoho metód analýzy biologických sekvencií je preto zakorenená v informatike, kde je už mnoho literatúry týkajúcej sa tejto problematiky.

Vývoj sekvencií hromadí \textit{inzercie}, \textit{delécie} a \textit{substitúcie}, takže predtým ako môže byť vyhodnotená podobnosť, treba urobiť zarovnanie sekvencií. Preto je zarovnanie sekvencií veľmi dôležité.

\section{Skórovacie schémy}
Takmer všetky metódy zarovnania hľadajú zarovnanie dvoch reťazcov na základe nejakej \textit{skórovacej schémy}. Skórovacie schémy môžu byť veľmi jednoduché, napr. $+1$ za \textit{zhodu} a $-1$ za \textit{nezhodu}. Hoci ak chceme mať schému, kde biologicky najkorektnjšie zarovnanie má najvyššie skóre, musíme vziať do úvahy, že biologické sekvencie majú evolučnú históriu, $3D$ štrukturu a mnohé ďalšie vlastnosti obmedzujúce ich primárnu evolúciu. Preto skórovací systém vyžaduje starostlivé premyslenie a môže byť veľmi zložitý.

\section{Párové zarovnávanie}
Párové zarovnávanie je základná úloha zarovnávania sekvencií, kde sa k sebe zarovnávajú dve sekvencie. V tejto práci sa budeme zaoberať len párovým zarovnávaním.

Kľúčové problémy sú:
\begin{enumerate}
\item Aké typy zarovnávania by sme mali uvažovať
\item Skórovací systém, ktorý použijeme na ohodnotenie zarovnania
\item Algoritmus, ktorý použijeme na hľadanie optimálneho alebo dobrého zarovnania podľa skórovacieho systému
\item Štatistické metódy na vyhodnotenie významnosti skóre zarovnania.
\end{enumerate}

\subsection{Typy zarovnaní}
Základné typy zarovnaní sú \textit{Globálne zarovnanie} a \textit{Lokálne zarovnanie}.\\ 
Pri globálnom zarovnaní je výstupom zarovnanie celých sekvencií s najvyšším skóre.\\
Pri lokálnomn zarovnaní je výstupom zarovnanie nejakých poreťazcov sekvencií s najvyšším skóre.

%\subsection{Skórovacie systémy}

\subsection{Algoritmy na hľadanie zarovnaní}
Máme daný skórovací systém, potrebujeme algoritmus, ktorý nájde optimálne zarovnanie dvoch sekvencií.
Budeme uvažovať zarovnávanie s medzerami. To znamená, že môžeme do sekvencie pridať ľubovoľne veľa medzier, aby sme dosiahli lepšie skóre. Pre 2 sekvencie dĺžky $n$ existuje 
$$ {2n \choose n}  = \frac{(2n)!}{(n!)^2} \simeq \frac{2^{2n}}{\sqrt{\pi n}} $$
možných globálnych zarovnaní. Čiže nie je možné v rozumnom čase nimi prejsť.

Algoritmy na hľadanie zarovnaní využívajú \textit{dynamické programovanie}. Algoritmy s dynamickým programovaním garantujú nájdenie optimálneho zarovnania.
Existujú aj heuristické algoritmy, ktoré môžu byť veľmi rýchle, avšak majú určité predpoklady a môže sa stať, že nenájdu najlepšie zarovnanie pre niektoré páry sekvencií.
My sa budeme zaoberať len algoritmami využívajúcimi dynamické programovanie. Pre rôzne typy zarovnaní máme rôzne algoritmy zarovnávania.

\subsection{Algoritmus pre globálne zarovnanie: Needelman-Wunch}
Máme dané 2 sekvencie $X = x_1x_2\dots x_n$ a $Y = y_1y_2\dots y_n$, budeme zarovnávať všetky znaky sekvencie $X$ a všetky znaky sekvencie $Y$. Budeme používať jednoduché skórovanie: $+1$ za zhodu, $-1$ za nezhodu alebo pomlčku.

%Idea je nasledovná - Nech máme optimálne zarovnanie dĺžky $n$. Zarovnanie dĺžky $n+1$ vieme vyrobiť 3 spôsobmi. Buď pridáme $(x_i,y_j)$, alebo $(x_i,-)$, alebo $(-,y_j)$, kde $i,\ j$ sú indexy prvej nepoužitej bázy v $X$ resp. $Y$. Možnosť s najlepším skore nám dá optimálne zarovnanie dĺžky $n+1$.

Algoritmus postupne vypĺňa 2-rozmernú tabuľku $A$. Riadky zodpovedajú bázam sekvencie $X$ a stĺpce bázam $Y$. Na políčku $A[i,j]$ bude skóre najlepšieho zarovnania prvých $i$ báz sekvencie $X$ a prvých $j$ báz $Y$.

Ak zarovnávame sekvenciu s prázdnou sekvenciou, tak skóre bude $-n$, kde $n$ je dĺžka sekvencie. Bude tam $n$ pomlčiek, každá nám dá skóre $-1$. Takto vyplníme riadky a stĺpce $A[i,0]$ a $A[0,j]$.

Ak chceme vyplniť políčko $A[i,j]$, musíme si uvedomiť ako môže vyzerať posledný stĺpec zarovnania $x_1x_2\dots x_i$ a $y_1y_2\dots y_j$. Máme iba 3 možnosti ako môže vyzerať posledný stĺpec najlepšieho zarovnania. Buď obsahuje $x_i$ alebo $y_j$ alebo oboje. Ak posledný stĺpec obsahuje oboje, cena tohto stĺpca je buď $+1$ ak $x_i = y_j$ alebo $-1$ ináč. AK by sme posledný stĺpec zmazali, dostali by sme zarovnanie $x_1x_2\dots x_{i-1}$ a $y_1y_2\dots y_{j-1}$, pričom musí ísť o najlepšie zarovnanie. To už máme vypočítané v políčku $A[i-1, j-1]$, čiže výsledné skóre bude $A[i-1, j-1] + s(x_i,y_j)$.

Ak posledný stĺpec obsahuje len $x_i$ zarovnané s pomlčkou, skóre stĺpca bude $-1$ a po zmazaní dostávame zarovnanie $x_1x_2\dots x_{i-1}$ a $y_1y_2\dots y_{j}$, výsledné skóre bude teda $A[i-1, j] -1$. V prípade, že posledný stĺpec obsahuje len $y_i$, tak skóre vypočítame analogicky.

Najlepšie skóre bude maximum skóre pre všetky 3 prípady.
Dostávame teda nasledujúci vzťah pre výpočet $A[i,j]$:
$$A[i,j] = \max \left\{ 
\begin{array}{l}
A[i-1,j-1]+s(x_i, y_j)\\ 
A[i-1,j]-1\\
A[i,j-1]-1 
\end{array} \right.$$
Tabuľku vieme vypĺňať po riadkoch, pričom každé políčok vieme vypočítať z troch políčok, ktoré už máme vypočítané. 

Ak nás zaujíma aj zarovnanie -- nie len jeho skóre -- vieme si pre každé políčko zapamätať ktorá z 3 možností dosiahla maximálnu hodnotu. Na základe tejto informácie potom vieme zrekonštruovať zarovnanie tak, že postupne z posledného políčka ($A[n,m]$) budeme prechádzať na políčko, z ktorého sme vypočítali aktuálnu hodnotu.

Časová zložitosť je $O(nm)$, pretože vypĺňame $nm$ políčok, každé v konštantnom čase. Zjavne aj pamäťová zložitosť je $O(nm)$.

Pamäťová zložitosť sa dá zredukovať na $O(n+m)$ za cenu zhruba dvojnásobného času výpočtu \cite{hirschberg}.

\subsection{Algoritmus pre lokálne zarovnanie: Smith-Waterman}

% TODO: preformulovať aby sedelo na lokalne zarovnania
%Máme dané 2 sekvencie $X = x_1x_2\dots x_n$ a $Y = y_1y_2\dots y_n$, budeme zarovnávať všetky znaky sekvencie $X$ a všetky znaky sekvencie $Y$. Budeme používať jednoduché skórovanie: $+1$ za zhodu, $-1$ za nezhodu alebo pomlčku. 

Algoritmus pre lokálne zarovnania sa líši len v niekoľkých malých detajloch. Opäť vypĺňame tabuľku $A$, s tým, že v $A[i,j]$ bude najvyššie skóre lokálneho zarovnania medzi sekvenciami $x_1x_2\dots x_i$ a $y_1y_2\dots y_j$, ktoré buď obsahuje bázy $x_i$ aj $y_j$, alebo je prázdne. Teda na ľubovoľnom mieste uvažujeme aj prázdne zarovnanie so skóre 0 (v matici nebudú záporné čísla). Vzťah pre výpočet $A[i,j]$ vyzerá takto:
$$A[i,j] = \max \left\{ 
\begin{array}{l}
0\\
A[i-1,j-1]+s(x_i, y_j)\\ 
A[i-1,j]-1\\
A[i,j-1]-1 
\end{array} \right.$$

Časová aj pamäťová zložitosť sú, rovnako ako pri globálnom zarovnaní $O(nm)$.